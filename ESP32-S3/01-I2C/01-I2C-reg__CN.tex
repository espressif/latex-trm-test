\begin{register}{H}{I2C\_SCL\_LOW\_PERIOD\_REG}{0x{}0000}\label{regdesc:I2CSCLLOWPERIODREG}
\regfield{(reserved)}{23}{9}{00000000000000000000000}%
\regfield{I2C\_SCL\_LOW\_PERIOD}{9}{0}{{0}}%
\reglabel{Reset}\regnewline%
\begin{regdesc}\begin{reglist}
\label{fielddesc:I2CSCLLOWPERIOD}\item [I2C\_SCL\_LOW\_PERIOD] 用于配置主机模式下 SCL 低电平的保持时间,以 I2C 控制器时钟周期数为单位。 (R/W)
\end{reglist}\end{regdesc}
\end{register}


\begin{register}{H}{I2C\_SDA\_HOLD\_REG}{0x{}0030}\label{regdesc:I2CSDAHOLDREG}
\regfield{(reserved)}{23}{9}{00000000000000000000000}%
\regfield{I2C\_SDA\_HOLD\_TIME}{9}{0}{{0}}%
\reglabel{Reset}\regnewline%
\begin{regdesc}\begin{reglist}
\label{fielddesc:I2CSDAHOLDTIME}\item [I2C\_SDA\_HOLD\_TIME] 用于配置 SCL 下降沿后的数据保持时间,以 I2C 控制器时钟周期数为单位。 (R/W)
\end{reglist}\end{regdesc}
\end{register}


\begin{register}{H}{I2C\_SDA\_SAMPLE\_REG}{0x{}0034}\label{regdesc:I2CSDASAMPLEREG}
\regfield{(reserved)}{23}{9}{00000000000000000000000}%
\regfield{I2C\_SDA\_SAMPLE\_TIME}{9}{0}{{0}}%
\reglabel{Reset}\regnewline%
\begin{regdesc}\begin{reglist}
\label{fielddesc:I2CSDASAMPLETIME}\item [I2C\_SDA\_SAMPLE\_TIME] 用于配置采样 SDA 的时间,以 I2C 控制器时钟周期数为单位。 (R/W)
\end{reglist}\end{regdesc}
\end{register}


\begin{register}{H}{I2C\_SCL\_HIGH\_PERIOD\_REG}{0x{}0038}\label{regdesc:I2CSCLHIGHPERIODREG}
\regfield{(reserved)}{16}{16}{0000000000000000}%
\regfield{I2C\_SCL\_WAIT\_HIGH\_PERIOD}{7}{9}{{0}}%
\regfield{I2C\_SCL\_HIGH\_PERIOD}{9}{0}{{0}}%
\reglabel{Reset}\regnewline%
\begin{regdesc}\begin{reglist}
\label{fielddesc:I2CSCLHIGHPERIOD}\item [I2C\_SCL\_HIGH\_PERIOD] 用于配置 SCL 在主机模式下保持高电平的时间,以 I2C 控制器时钟周期数为单位。 (R/W)
\label{fielddesc:I2CSCLWAITHIGHPERIOD}\item [I2C\_SCL\_WAIT\_HIGH\_PERIOD] 用于配置 SCL\_FSM 等待 SCL 在主机模式下翻转至高电平的时间,以 I2C 控制器时钟周期数为单位。 (R/W)
\end{reglist}\end{regdesc}
\end{register}


\begin{register}{H}{I2C\_SCL\_START\_HOLD\_REG}{0x{}0040}\label{regdesc:I2CSCLSTARTHOLDREG}
\regfield{(reserved)}{23}{9}{00000000000000000000000}%
\regfield{I2C\_SCL\_START\_HOLD\_TIME}{9}{0}{{8}}%
\reglabel{Reset}\regnewline%
\begin{regdesc}\begin{reglist}
\label{fielddesc:I2CSCLSTARTHOLDTIME}\item [I2C\_SCL\_START\_HOLD\_TIME] 配置 START 命令产生时 SDA 下降沿 和 SCL 下降沿的间隔时间,以 I2C 控制器时钟周期数为单位。 (R/W)
\end{reglist}\end{regdesc}
\end{register}


\begin{register}{H}{I2C\_SCL\_RSTART\_SETUP\_REG}{0x{}0044}\label{regdesc:I2CSCLRSTARTSETUPREG}
\regfield{(reserved)}{23}{9}{00000000000000000000000}%
\regfield{I2C\_SCL\_RSTART\_SETUP\_TIME}{9}{0}{{8}}%
\reglabel{Reset}\regnewline%
\begin{regdesc}\begin{reglist}
\label{fielddesc:I2CSCLRSTARTSETUPTIME}\item [I2C\_SCL\_RSTART\_SETUP\_TIME] 配置 RSTART 命令产生时 SCL 上升沿和 SDA 下降沿的间隔时间,以 I2C 控制器的时钟周期数为单位。 (R/W)
\end{reglist}\end{regdesc}
\end{register}


\begin{register}{H}{I2C\_SCL\_STOP\_HOLD\_REG}{0x{}0048}\label{regdesc:I2CSCLSTOPHOLDREG}
\regfield{(reserved)}{23}{9}{00000000000000000000000}%
\regfield{I2C\_SCL\_STOP\_HOLD\_TIME}{9}{0}{{8}}%
\reglabel{Reset}\regnewline%
\begin{regdesc}\begin{reglist}
\label{fielddesc:I2CSCLSTOPHOLDTIME}\item [I2C\_SCL\_STOP\_HOLD\_TIME] 配置 STOP 命令后的延迟,以 I2C 控制器时钟周期数为单位。 (R/W)
\end{reglist}\end{regdesc}
\end{register}


\begin{register}{H}{I2C\_SCL\_STOP\_SETUP\_REG}{0x{}004C}\label{regdesc:I2CSCLSTOPSETUPREG}
\regfield{(reserved)}{23}{9}{00000000000000000000000}%
\regfield{I2C\_SCL\_STOP\_SETUP\_TIME}{9}{0}{{8}}%
\reglabel{Reset}\regnewline%
\begin{regdesc}\begin{reglist}
\label{fielddesc:I2CSCLSTOPSETUPTIME}\item [I2C\_SCL\_STOP\_SETUP\_TIME] 配置 SCL 上升沿和 SDA 上升沿的间隔时间,以 I2C 控制器时钟周期数为单位。 (R/W)
\end{reglist}\end{regdesc}
\end{register}


\begin{register}{H}{I2C\_SCL\_ST\_TIME\_OUT\_REG}{0x{}0078}\label{regdesc:I2CSCLSTTIMEOUTREG}
\regfield{(reserved)}{27}{5}{000000000000000000000000000}%
\regfield{I2C\_SCL\_ST\_TO\_I2C}{5}{0}{{0x{}10}}%
\reglabel{Reset}\regnewline%
\begin{regdesc}\begin{reglist}
\label{fielddesc:I2CSCLSTTOI2C}\item [I2C\_SCL\_ST\_TO\_I2C] SCL\_FSM 状态不变的最大时间,不能大于 23。 (R/W)
\end{reglist}\end{regdesc}
\end{register}


\begin{register}{H}{I2C\_SCL\_MAIN\_ST\_TIME\_OUT\_REG}{0x{}007C}\label{regdesc:I2CSCLMAINSTTIMEOUTREG}
\regfield{(reserved)}{27}{5}{000000000000000000000000000}%
\regfield{I2C\_SCL\_MAIN\_ST\_TO\_I2C}{5}{0}{{0x{}10}}%
\reglabel{Reset}\regnewline%
\begin{regdesc}\begin{reglist}
\label{fielddesc:I2CSCLMAINSTTOI2C}\item [I2C\_SCL\_MAIN\_ST\_TO\_I2C] SCL\_MAIN\_FSM 状态不变的最大时间,不能大于 23。 (R/W)
\end{reglist}\end{regdesc}
\end{register}


\begin{register}{H}{I2C\_CTR\_REG}{0x{}0004}\label{regdesc:I2CCTRREG}
\regfield{(reserved)}{17}{15}{00000000000000000}%
\regfield{I2C\_ADDR\_BROADCASTING\_EN}{1}{14}{{0}}%
\regfield{I2C\_ADDR\_10BIT\_RW\_CHECK\_EN}{1}{13}{{0}}%
\regfield{I2C\_SLV\_TX\_AUTO\_START\_EN}{1}{12}{{0}}%
\regfield{I2C\_CONF\_UPGATE}{1}{11}{{0}}%
\regfield{I2C\_FSM\_RST}{1}{10}{{0}}%
\regfield{I2C\_ARBITRATION\_EN}{1}{9}{{1}}%
\regfield{I2C\_CLK\_EN}{1}{8}{{0}}%
\regfield{I2C\_RX\_LSB\_FIRST}{1}{7}{{0}}%
\regfield{I2C\_TX\_LSB\_FIRST}{1}{6}{{0}}%
\regfield{I2C\_TRANS\_START}{1}{5}{{0}}%
\regfield{I2C\_MS\_MODE}{1}{4}{{0}}%
\regfield{I2C\_RX\_FULL\_ACK\_LEVEL}{1}{3}{{1}}%
\regfield{I2C\_SAMPLE\_SCL\_LEVEL}{1}{2}{{0}}%
\regfield{I2C\_SCL\_FORCE\_OUT}{1}{1}{{1}}%
\regfield{I2C\_SDA\_FORCE\_OUT}{1}{0}{{1}}%
\reglabel{Reset}\regnewline%
\begin{regdesc}\begin{reglist}
\label{fielddesc:I2CSDAFORCEOUT}\item [I2C\_SDA\_FORCE\_OUT] 配置 SDA 输出模式。\\
0: 开漏输出 \\ 
1: 直接输出 \\ (R/W)
\label{fielddesc:I2CSCLFORCEOUT}\item [I2C\_SCL\_FORCE\_OUT] 配置 SCL 输出模式。\\
0: 开漏输出 \\ 
1: 直接输出 \\ (R/W)
\label{fielddesc:I2CSAMPLESCLLEVEL}\item [I2C\_SAMPLE\_SCL\_LEVEL] 用于选择采样模式。
0:SCL 为高电平时采样 SDA 数据;
1:SCL 为低电平时采样 SDA 数据。 (R/W)
\label{fielddesc:I2CRXFULLACKLEVEL}\item [I2C\_RX\_FULL\_ACK\_LEVEL] 用于配置主机在 I2C\_RXFIFO\_CNT 达到阈值时需发送的 ACK 电平值。 (R/W)
\label{fielddesc:I2CMSMODE}\item [I2C\_MS\_MODE] 置位此位,将 I2C 控制器配置为主机。 清零此位,将 I2C 控制器配置为从机。
 (R/W)
\label{fielddesc:I2CTRANSSTART}\item [I2C\_TRANS\_START] 置位此位,开始发送 TX FIFO 中的数据。 (WT)
\label{fielddesc:I2CTXLSBFIRST}\item [I2C\_TX\_LSB\_FIRST] 用于控制待发送数据的发送顺序。
0:从最高有效位开始发送数据;
1:从最低有效位开始发送数据。 (R/W)
\label{fielddesc:I2CRXLSBFIRST}\item [I2C\_RX\_LSB\_FIRST] 用于控制接收数据的存储顺序。
0:从最高有效位开始接收数据;
1:从最低有效位开始接收数据。 (R/W)
\label{fielddesc:I2CCLKEN}\item [I2C\_CLK\_EN] 用于控制 APB\_CLK 时钟门控。0:APB\_CLK 时钟门控使能,以便节能;1:APB\_CLK 时钟一直开启。 (R/W)
\label{fielddesc:I2CARBITRATIONEN}\item [I2C\_ARBITRATION\_EN] I2C 总线仲裁的使能位。 (R/W)
\label{fielddesc:I2CFSMRST}\item [I2C\_FSM\_RST] 用于复位 SCL\_FSM。 (WT)
\label{fielddesc:I2CCONFUPGATE}\item [I2C\_CONF\_UPGATE] 同步位。 (WT)
\label{fielddesc:I2CSLVTXAUTOSTARTEN}\item [I2C\_SLV\_TX\_AUTO\_START\_EN] 从机自动发送数据的使能位。 (R/W)
\label{fielddesc:I2CADDR10BITRWCHECKEN}\item [I2C\_ADDR\_10BIT\_RW\_CHECK\_EN] 使能 10 位寻址模式的读写标志位检查功能,检查读写标志是否符合协议。 (R/W)
\label{fielddesc:I2CADDRBROADCASTINGEN}\item [I2C\_ADDR\_BROADCASTING\_EN] 使能 7 位寻址模式的广播功能。 (R/W)
\end{reglist}\end{regdesc}
\end{register}


\begin{register}{H}{I2C\_TO\_REG}{0x{}000C}\label{regdesc:I2CTOREG}
\regfield{(reserved)}{26}{6}{00000000000000000000000000}%
\regfield{I2C\_TIME\_OUT\_EN}{1}{5}{{0}}%
\regfield{I2C\_TIME\_OUT\_VALUE}{5}{0}{{0x{}10}}%
\reglabel{Reset}\regnewline%
\begin{regdesc}\begin{reglist}
\label{fielddesc:I2CTIMEOUTVALUE}\item [I2C\_TIME\_OUT\_VALUE] 用于配置接收一位数据的超时时间,以 I2C\_SCLK 时钟周期为单位。配置的超时时间为  2$^{I2C\_TIME\_OUT\_VALUE}$ 个时钟周期。 (R/W)
\label{fielddesc:I2CTIMEOUTEN}\item [I2C\_TIME\_OUT\_EN] 超时控制使能位。 (R/W)
\end{reglist}\end{regdesc}
\end{register}


\begin{register}{H}{I2C\_SLAVE\_ADDR\_REG}{0x{}0010}\label{regdesc:I2CSLAVEADDRREG}
\regfield{I2C\_ADDR\_10BIT\_EN}{1}{31}{{0}}%
\regfield{(reserved)}{16}{15}{0000000000000000}%
\regfield{I2C\_SLAVE\_ADDR}{15}{0}{{0}}%
\reglabel{Reset}\regnewline%
\begin{regdesc}\begin{reglist}
\label{fielddesc:I2CSLAVEADDR}\item [I2C\_SLAVE\_ADDR] I2C 控制器配置为从机 时,该字段用于配置从机地址。 (R/W)
\label{fielddesc:I2CADDR10BITEN}\item [I2C\_ADDR\_10BIT\_EN] 用于在主机模式下使能从机的 10 位寻址模式。 (R/W)
\end{reglist}\end{regdesc}
\end{register}


\begin{register}{H}{I2C\_FIFO\_CONF\_REG}{0x{}0018}\label{regdesc:I2CFIFOCONFREG}
\regfield{(reserved)}{17}{15}{00000000000000000}%
\regfield{I2C\_FIFO\_PRT\_EN}{1}{14}{{1}}%
\regfield{I2C\_TX\_FIFO\_RST}{1}{13}{{0}}%
\regfield{I2C\_RX\_FIFO\_RST}{1}{12}{{0}}%
\regfield{I2C\_FIFO\_ADDR\_CFG\_EN}{1}{11}{{0}}%
\regfield{I2C\_NONFIFO\_EN}{1}{10}{{0}}%
\regfield{I2C\_TXFIFO\_WM\_THRHD}{5}{5}{{0x{}4}}%
\regfield{I2C\_RXFIFO\_WM\_THRHD}{5}{0}{{0x{}b}}%
\reglabel{Reset}\regnewline%
\begin{regdesc}\begin{reglist}
\label{fielddesc:I2CRXFIFOWMTHRHD}\item [I2C\_RXFIFO\_WM\_THRHD] 直接访问模式下,RX FIFO 的水标阈值。I2C\_FIFO\_PRT\_EN 为 1 且 RX FIFO 计数值大于 I2C\_TXFIFO\_WM\_THRHD[4:0] 时,I2C\_TXFIFO\_WM\_INT\_RAW 位有效。 (R/W)
\label{fielddesc:I2CTXFIFOWMTHRHD}\item [I2C\_TXFIFO\_WM\_THRHD] 直接访问模式下,TX FIFO 的水标阈值。I2C\_FIFO\_PRT\_EN 为 1 且 TX FIFO 计数值小于 I2C\_TXFIFO\_WM\_THRHD[4:0] 时,I2C\_TXFIFO\_WM\_INT\_RAW 位有效。 (R/W)
\label{fielddesc:I2CNONFIFOEN}\item [I2C\_NONFIFO\_EN] 置位此位,使能 APB 直接访问。 (R/W)
\label{fielddesc:I2CFIFOADDRCFGEN}\item [I2C\_FIFO\_ADDR\_CFG\_EN] 此位置 1 时,从机接收地址字节的后一个字节为从机 RAM 中的偏移地址。 (R/W)
\label{fielddesc:I2CRXFIFORST}\item [I2C\_RX\_FIFO\_RST] 置位此位,复位 RX FIFO 。 (R/W)
\label{fielddesc:I2CTXFIFORST}\item [I2C\_TX\_FIFO\_RST] 置位此位,复位 TX FIFO 。 (R/W)
\label{fielddesc:I2CFIFOPRTEN}\item [I2C\_FIFO\_PRT\_EN] 直接访问模式下 FIFO 指针的控制使能位。 该位控制 TX FIFO 和 RX FIFO 溢出、下溢、为满、为空时的有效位和中断。 (R/W)
\end{reglist}\end{regdesc}
\end{register}


\begin{register}{H}{I2C\_FILTER\_CFG\_REG}{0x{}0050}\label{regdesc:I2CFILTERCFGREG}
\regfield{(reserved)}{22}{10}{0000000000000000000000}%
\regfield{I2C\_SDA\_FILTER\_EN}{1}{9}{{1}}%
\regfield{I2C\_SCL\_FILTER\_EN}{1}{8}{{1}}%
\regfield{I2C\_SDA\_FILTER\_THRES}{4}{4}{{0}}%
\regfield{I2C\_SCL\_FILTER\_THRES}{4}{0}{{0}}%
\reglabel{Reset}\regnewline%
\begin{regdesc}\begin{reglist}
\label{fielddesc:I2CSCLFILTERTHRES}\item [I2C\_SCL\_FILTER\_THRES] SCL 输入信号的脉冲宽度小于该字段的值时,I2C 控制器忽略此脉冲。该寄存器的值以 I2C 控制器时钟周期数为单位。 (R/W)
\label{fielddesc:I2CSDAFILTERTHRES}\item [I2C\_SDA\_FILTER\_THRES] SDA 输入信号的脉冲宽度小于该字段的值时,I2C 控制器忽略此脉冲。该寄存器的值以 I2C 控制器时钟周期数为单位。 (R/W)
\label{fielddesc:I2CSCLFILTEREN}\item [I2C\_SCL\_FILTER\_EN] SCL 的滤波使能位。 (R/W)
\label{fielddesc:I2CSDAFILTEREN}\item [I2C\_SDA\_FILTER\_EN] SDA 的滤波使能位。 (R/W)
\end{reglist}\end{regdesc}
\end{register}


\begin{register}{H}{I2C\_CLK\_CONF\_REG}{0x{}0054}\label{regdesc:I2CCLKCONFREG}
\regfield{(reserved)}{10}{22}{0000000000}%
\regfield{I2C\_SCLK\_ACTIVE}{1}{21}{{1}}%
\regfield{I2C\_SCLK\_SEL}{1}{20}{{0}}%
\regfield{I2C\_SCLK\_DIV\_B}{6}{14}{{0}}%
\regfield{I2C\_SCLK\_DIV\_A}{6}{8}{{0}}%
\regfield{I2C\_SCLK\_DIV\_NUM}{8}{0}{{0}}%
\reglabel{Reset}\regnewline%
\begin{regdesc}\begin{reglist}
\label{fielddesc:I2CSCLKDIVNUM}\item [I2C\_SCLK\_DIV\_NUM] 分频系数的整数部分。 (R/W)
\label{fielddesc:I2CSCLKDIVA}\item [I2C\_SCLK\_DIV\_A] 分频系数小数部分的分子。 (R/W)
\label{fielddesc:I2CSCLKDIVB}\item [I2C\_SCLK\_DIV\_B] 分频系数小数部分的分母。 (R/W)
\label{fielddesc:I2CSCLKSEL}\item [I2C\_SCLK\_SEL] 选择 I2C 控制器的时钟源。0:XTAL\_CLK;1:RC\_FAST\_CLK。 (R/W)
\label{fielddesc:I2CSCLKACTIVE}\item [I2C\_SCLK\_ACTIVE] I2C 控制器的时钟开关。 (R/W)
\end{reglist}\end{regdesc}
\end{register}


\begin{register}{H}{I2C\_SCL\_SP\_CONF\_REG}{0x{}0080}\label{regdesc:I2CSCLSPCONFREG}
\regfield{(reserved)}{24}{8}{000000000000000000000000}%
\regfield{I2C\_SDA\_PD\_EN}{1}{7}{{0}}%
\regfield{I2C\_SCL\_PD\_EN}{1}{6}{{0}}%
\regfield{I2C\_SCL\_RST\_SLV\_NUM}{5}{1}{{0}}%
\regfield{I2C\_SCL\_RST\_SLV\_EN}{1}{0}{{0}}%
\reglabel{Reset}\regnewline%
\begin{regdesc}\begin{reglist}
\label{fielddesc:I2CSCLRSTSLVEN}\item [I2C\_SCL\_RST\_SLV\_EN] I2C 主机处于空闲状态时,置位此位发送 SCL 脉冲。 脉冲数量为 I2C\_SCL\_RST\_SLV\_NUM[4:0]。 (R/W/SC)
\label{fielddesc:I2CSCLRSTSLVNUM}\item [I2C\_SCL\_RST\_SLV\_NUM] 配置主机模式下生成的 SCL 脉冲。I2C\_SCL\_RST\_SLV\_EN 为 1 时有效。 (R/W)
\label{fielddesc:I2CSCLPDEN}\item [I2C\_SCL\_PD\_EN] 降低 I2C SCL 输出功耗的使能位。 0:正常工作;1:不工作,降低功耗。将 I2C\_SCL\_FORCE\_OUT 和 I2C\_SCL\_PD\_EN 置 1 拉伸 SCL。 (R/W)
\label{fielddesc:I2CSDAPDEN}\item [I2C\_SDA\_PD\_EN] 降低 I2C SDA 输出功耗的使能位。0:正常工作;1:不工作,降低功耗。将 I2C\_SDA\_FORCE\_OUT 和 I2C\_SDA\_PD\_EN 置 1 拉伸 SDA 。 (R/W)
\end{reglist}\end{regdesc}
\end{register}


\begin{register}{H}{I2C\_SCL\_STRETCH\_CONF\_REG}{0x{}0084}\label{regdesc:I2CSCLSTRETCHCONFREG}
\regfield{(reserved)}{18}{14}{000000000000000000}%
\regfield{I2C\_SLAVE\_BYTE\_ACK\_LVL}{1}{13}{{0}}%
\regfield{I2C\_SLAVE\_BYTE\_ACK\_CTL\_EN}{1}{12}{{0}}%
\regfield{I2C\_SLAVE\_SCL\_STRETCH\_CLR}{1}{11}{{0}}%
\regfield{I2C\_SLAVE\_SCL\_STRETCH\_EN}{1}{10}{{0}}%
\regfield{I2C\_STRETCH\_PROTECT\_NUM}{10}{0}{{0}}%
\reglabel{Reset}\regnewline%
\begin{regdesc}\begin{reglist}
\label{fielddesc:I2CSTRETCHPROTECTNUM}\item [I2C\_STRETCH\_PROTECT\_NUM] 配置 SCL 时钟拉伸释放后的保护时间,通常设置为大于 SDA 的建立时间即可。 (R/W)
\label{fielddesc:I2CSLAVESCLSTRETCHEN}\item [I2C\_SLAVE\_SCL\_STRETCH\_EN]  SCL 时钟拉伸的使能位。 0:关闭;1:使能。I2C\_SLAVE\_SCL\_STRETCH\_EN 为 1,且出现可触发时钟拉伸的事件时,拉伸 SCL 时钟。 SCL 时钟拉伸的原因可见 I2C\_STRETCH\_CAUSE。 (R/W)
\label{fielddesc:I2CSLAVESCLSTRETCHCLR}\item [I2C\_SLAVE\_SCL\_STRETCH\_CLR] 置位此位,清除 CL 时钟拉伸。 (WT)
\label{fielddesc:I2CSLAVEBYTEACKCTLEN}\item [I2C\_SLAVE\_BYTE\_ACK\_CTL\_EN] 使能从机控制 ACK 电平。 (R/W)
\label{fielddesc:I2CSLAVEBYTEACKLVL}\item [I2C\_SLAVE\_BYTE\_ACK\_LVL] I2C\_SLAVE\_BYTE\_ACK\_CTL\_EN 置 1 时,设置 ACK 的电平值。 (R/W)
\end{reglist}\end{regdesc}
\end{register}


\begin{register}{H}{I2C\_SR\_REG}{0x{}0008}\label{regdesc:I2CSRREG}
\regfield{(reserved)}{1}{31}{0}%
\regfield{I2C\_SCL\_STATE\_LAST}{3}{28}{{0}}%
\regfield{(reserved)}{1}{27}{0}%
\regfield{I2C\_SCL\_MAIN\_STATE\_LAST}{3}{24}{{0}}%
\regfield{I2C\_TXFIFO\_CNT}{6}{18}{{0}}%
\regfield{(reserved)}{2}{16}{00}%
\regfield{I2C\_STRETCH\_CAUSE}{2}{14}{{0x{}3}}%
\regfield{I2C\_RXFIFO\_CNT}{6}{8}{{0}}%
\regfield{(reserved)}{2}{6}{00}%
\regfield{I2C\_SLAVE\_ADDRESSED}{1}{5}{{0}}%
\regfield{I2C\_BUS\_BUSY}{1}{4}{{0}}%
\regfield{I2C\_ARB\_LOST}{1}{3}{{0}}%
\regfield{(reserved)}{1}{2}{0}%
\regfield{I2C\_SLAVE\_RW}{1}{1}{{0}}%
\regfield{I2C\_RESP\_REC}{1}{0}{{0}}%
\reglabel{Reset}\regnewline%
\begin{regdesc}\begin{reglist}
\label{fielddesc:I2CRESPREC}\item [I2C\_RESP\_REC] 主机模式或从机模式下接收的 ACK 电平值。 0:ACK;1:NACK。 (RO)
\label{fielddesc:I2CSLAVERW}\item [I2C\_SLAVE\_RW] 从机模式下,0:主机向从机写入数据;1:主机读取从机数据。 (RO)
\label{fielddesc:I2CARBLOST}\item [I2C\_ARB\_LOST] I2C 控制器不控制 SCL 线时,该寄存器变为 1。 (RO)
\label{fielddesc:I2CBUSBUSY}\item [I2C\_BUS\_BUSY] 0:I2C 总线处于空闲状态;1:I2C 总线正在传输数据。 (RO)
\label{fielddesc:I2CSLAVEADDRESSED}\item [I2C\_SLAVE\_ADDRESSED] 配置成 I2C 从机、且主机发送地址与从机地址匹配时,该位翻转为高电平。 (RO)
\label{fielddesc:I2CRXFIFOCNT}\item [I2C\_RXFIFO\_CNT] 该字段为需发送数据的字节数。 (RO)
\label{fielddesc:I2CSTRETCHCAUSE}\item [I2C\_STRETCH\_CAUSE] 从机模式下 SCL 时钟拉伸的原因。0:主机开始读取数据时拉伸 SCL 时钟;1:从机模式下 I2C TX FIFO 读空时拉伸 SCL 时钟;2:从机模式下 I2C RX FIFO 写满时拉伸 SCL 时钟。 (RO)
\label{fielddesc:I2CTXFIFOCNT}\item [I2C\_TXFIFO\_CNT] 该字段存储 RAM 接收数据的字节数。 (RO)
\label{fielddesc:I2CSCLMAINSTATELAST}\item [I2C\_SCL\_MAIN\_STATE\_LAST] 该字段为 I2C 控制器状态机的状态。
0:空闲;1:地址偏移;2:ACK 地址;3:接收数据;4:发送数据;5:发送 ACK;6:等待 ACK (RO)
\label{fielddesc:I2CSCLSTATELAST}\item [I2C\_SCL\_STATE\_LAST] 该字段为生成 SCL 的状态机状态。
0:空闲状态;1:开始;2:下降沿;3:低电平;4:上升沿;5:高电平;6:停止 (RO)
\end{reglist}\end{regdesc}
\end{register}


\begin{register}{H}{I2C\_FIFO\_ST\_REG}{0x{}0014}\label{regdesc:I2CFIFOSTREG}
\regfield{(reserved)}{2}{30}{00}%
\regfield{I2C\_SLAVE\_RW\_POINT}{8}{22}{{0}}%
\regfield{(reserved)}{2}{20}{00}%
\regfield{I2C\_TXFIFO\_WADDR}{5}{15}{{0}}%
\regfield{I2C\_TXFIFO\_RADDR}{5}{10}{{0}}%
\regfield{I2C\_RXFIFO\_WADDR}{5}{5}{{0}}%
\regfield{I2C\_RXFIFO\_RADDR}{5}{0}{{0}}%
\reglabel{Reset}\regnewline%
\begin{regdesc}\begin{reglist}
\label{fielddesc:I2CRXFIFORADDR}\item [I2C\_RXFIFO\_RADDR] APB 总线读 RX FIFO 的偏移地址。 (RO)
\label{fielddesc:I2CRXFIFOWADDR}\item [I2C\_RXFIFO\_WADDR] I2C 控制器接收数据和写 RX FIFO 的偏移地址。 (RO)
\label{fielddesc:I2CTXFIFORADDR}\item [I2C\_TXFIFO\_RADDR] I2C 控制器读 TX FIFO 的偏移地址。 (RO)
\label{fielddesc:I2CTXFIFOWADDR}\item [I2C\_TXFIFO\_WADDR] APB 总线写 TX FIFO 的偏移地址。 (RO)
\label{fielddesc:I2CSLAVERWPOINT}\item [I2C\_SLAVE\_RW\_POINT] 从机模式下接收的数据。 (RO)
\end{reglist}\end{regdesc}
\end{register}


\begin{register}{H}{I2C\_DATA\_REG}{0x{}001C}\label{regdesc:I2CDATAREG}
\regfield{(reserved)}{24}{8}{000000000000000000000000}%
\regfield{I2C\_FIFO\_RDATA}{8}{0}{{0}}%
\reglabel{Reset}\regnewline%
\begin{regdesc}\begin{reglist}
\label{fielddesc:I2CFIFORDATA}\item [I2C\_FIFO\_RDATA] 用于读取 RX FIFO 的数据,或向 TX FIFO 写数据。 (R/W)
\end{reglist}\end{regdesc}
\end{register}


\begin{register}{H}{I2C\_INT\_RAW\_REG}{0x{}0020}\label{regdesc:I2CINTRAWREG}
\regfield{(reserved)}{14}{18}{00000000000000}%
\regfield{I2C\_GENERAL\_CALL\_INT\_RAW}{1}{17}{{0}}%
\regfield{I2C\_SLAVE\_STRETCH\_INT\_RAW}{1}{16}{{0}}%
\regfield{I2C\_DET\_START\_INT\_RAW}{1}{15}{{0}}%
\regfield{I2C\_SCL\_MAIN\_ST\_TO\_INT\_RAW}{1}{14}{{0}}%
\regfield{I2C\_SCL\_ST\_TO\_INT\_RAW}{1}{13}{{0}}%
\regfield{I2C\_RXFIFO\_UDF\_INT\_RAW}{1}{12}{{0}}%
\regfield{I2C\_TXFIFO\_OVF\_INT\_RAW}{1}{11}{{0}}%
\regfield{I2C\_NACK\_INT\_RAW}{1}{10}{{0}}%
\regfield{I2C\_TRANS\_START\_INT\_RAW}{1}{9}{{0}}%
\regfield{I2C\_TIME\_OUT\_INT\_RAW}{1}{8}{{0}}%
\regfield{I2C\_TRANS\_COMPLETE\_INT\_RAW}{1}{7}{{0}}%
\regfield{I2C\_MST\_TXFIFO\_UDF\_INT\_RAW}{1}{6}{{0}}%
\regfield{I2C\_ARBITRATION\_LOST\_INT\_RAW}{1}{5}{{0}}%
\regfield{I2C\_BYTE\_TRANS\_DONE\_INT\_RAW}{1}{4}{{0}}%
\regfield{I2C\_END\_DETECT\_INT\_RAW}{1}{3}{{0}}%
\regfield{I2C\_RXFIFO\_OVF\_INT\_RAW}{1}{2}{{0}}%
\regfield{I2C\_TXFIFO\_WM\_INT\_RAW}{1}{1}{{1}}%
\regfield{I2C\_RXFIFO\_WM\_INT\_RAW}{1}{0}{{0}}%
\reglabel{Reset}\regnewline%
\begin{regdesc}\begin{reglist}
\label{fielddesc:I2CRXFIFOWMINTRAW}\item [I2C\_RXFIFO\_WM\_INT\_RAW] I2C\_RXFIFO\_WM\_INT 的原始中断位。 (R/SS/WTC)
\label{fielddesc:I2CTXFIFOWMINTRAW}\item [I2C\_TXFIFO\_WM\_INT\_RAW] I2C\_TXFIFO\_WM\_INT 的原始中断位。 (R/SS/WTC)
\label{fielddesc:I2CRXFIFOOVFINTRAW}\item [I2C\_RXFIFO\_OVF\_INT\_RAW] I2C\_RXFIFO\_OVF\_INT 的原始中断位。 (R/SS/WTC)
\label{fielddesc:I2CENDDETECTINTRAW}\item [I2C\_END\_DETECT\_INT\_RAW] I2C\_END\_DETECT\_INT 的原始中断位。 (R/SS/WTC)
\label{fielddesc:I2CBYTETRANSDONEINTRAW}\item [I2C\_BYTE\_TRANS\_DONE\_INT\_RAW] I2C\_BYTE\_TRANS\_DONE\_INT 的原始中断位。 (R/SS/WTC)
\label{fielddesc:I2CARBITRATIONLOSTINTRAW}\item [I2C\_ARBITRATION\_LOST\_INT\_RAW] I2C\_ARBITRATION\_LOST\_INT 的原始中断位。 (R/SS/WTC)
\label{fielddesc:I2CMSTTXFIFOUDFINTRAW}\item [I2C\_MST\_TXFIFO\_UDF\_INT\_RAW] I2C\_MST\_TXFIFO\_UDF\_INT 的原始中断位。 (R/SS/WTC)
\label{fielddesc:I2CTRANSCOMPLETEINTRAW}\item [I2C\_TRANS\_COMPLETE\_INT\_RAW] I2C\_TRANS\_COMPLETE\_INT 的原始中断位。 (R/SS/WTC)
\label{fielddesc:I2CTIMEOUTINTRAW}\item [I2C\_TIME\_OUT\_INT\_RAW] I2C\_TIME\_OUT\_INT 的原始中断位。 (R/SS/WTC)
\label{fielddesc:I2CTRANSSTARTINTRAW}\item [I2C\_TRANS\_START\_INT\_RAW] I2C\_TRANS\_START\_INT 的原始中断位。 (R/SS/WTC)
\label{fielddesc:I2CNACKINTRAW}\item [I2C\_NACK\_INT\_RAW] I2C\_NACK\_INT 的原始中断位。 (R/SS/WTC)
\label{fielddesc:I2CTXFIFOOVFINTRAW}\item [I2C\_TXFIFO\_OVF\_INT\_RAW] I2C\_TXFIFO\_OVF\_INT 的原始中断位。 (R/SS/WTC)
\label{fielddesc:I2CRXFIFOUDFINTRAW}\item [I2C\_RXFIFO\_UDF\_INT\_RAW] I2C\_RXFIFO\_UDF\_INT  的原始中断位。 (R/SS/WTC)
\label{fielddesc:I2CSCLSTTOINTRAW}\item [I2C\_SCL\_ST\_TO\_INT\_RAW] I2C\_SCL\_ST\_TO\_INT 的原始中断位。 (R/SS/WTC)
\label{fielddesc:I2CSCLMAINSTTOINTRAW}\item [I2C\_SCL\_MAIN\_ST\_TO\_INT\_RAW] I2C\_SCL\_MAIN\_ST\_TO\_INT 的原始中断位。 (R/SS/WTC)
\label{fielddesc:I2CDETSTARTINTRAW}\item [I2C\_DET\_START\_INT\_RAW] I2C\_DET\_START\_INT 的原始中断位。 (R/SS/WTC)
\label{fielddesc:I2CSLAVESTRETCHINTRAW}\item [I2C\_SLAVE\_STRETCH\_INT\_RAW] I2C\_SLAVE\_STRETCH\_INT 的原始中断位。 (R/SS/WTC)
\label{fielddesc:I2CGENERALCALLINTRAW}\item [I2C\_GENERAL\_CALL\_INT\_RAW] I2C\_GENARAL\_CALL\_INT 的原始中断位。 (R/SS/WTC)
\end{reglist}\end{regdesc}
\end{register}


\begin{register}{H}{I2C\_INT\_CLR\_REG}{0x{}0024}\label{regdesc:I2CINTCLRREG}
\regfield{(reserved)}{14}{18}{00000000000000}%
\regfield{I2C\_GENERAL\_CALL\_INT\_CLR}{1}{17}{{0}}%
\regfield{I2C\_SLAVE\_STRETCH\_INT\_CLR}{1}{16}{{0}}%
\regfield{I2C\_DET\_START\_INT\_CLR}{1}{15}{{0}}%
\regfield{I2C\_SCL\_MAIN\_ST\_TO\_INT\_CLR}{1}{14}{{0}}%
\regfield{I2C\_SCL\_ST\_TO\_INT\_CLR}{1}{13}{{0}}%
\regfield{I2C\_RXFIFO\_UDF\_INT\_CLR}{1}{12}{{0}}%
\regfield{I2C\_TXFIFO\_OVF\_INT\_CLR}{1}{11}{{0}}%
\regfield{I2C\_NACK\_INT\_CLR}{1}{10}{{0}}%
\regfield{I2C\_TRANS\_START\_INT\_CLR}{1}{9}{{0}}%
\regfield{I2C\_TIME\_OUT\_INT\_CLR}{1}{8}{{0}}%
\regfield{I2C\_TRANS\_COMPLETE\_INT\_CLR}{1}{7}{{0}}%
\regfield{I2C\_MST\_TXFIFO\_UDF\_INT\_CLR}{1}{6}{{0}}%
\regfield{I2C\_ARBITRATION\_LOST\_INT\_CLR}{1}{5}{{0}}%
\regfield{I2C\_BYTE\_TRANS\_DONE\_INT\_CLR}{1}{4}{{0}}%
\regfield{I2C\_END\_DETECT\_INT\_CLR}{1}{3}{{0}}%
\regfield{I2C\_RXFIFO\_OVF\_INT\_CLR}{1}{2}{{0}}%
\regfield{I2C\_TXFIFO\_WM\_INT\_CLR}{1}{1}{{0}}%
\regfield{I2C\_RXFIFO\_WM\_INT\_CLR}{1}{0}{{0}}%
\reglabel{Reset}\regnewline%
\begin{regdesc}\begin{reglist}
\label{fielddesc:I2CRXFIFOWMINTCLR}\item [I2C\_RXFIFO\_WM\_INT\_CLR] 置位此位,清除 I2C\_RXFIFO\_WM\_INT 中断。 (WT)
\label{fielddesc:I2CTXFIFOWMINTCLR}\item [I2C\_TXFIFO\_WM\_INT\_CLR] 置位此位,清除 I2C\_TXFIFO\_WM\_INT 中断。 (WT)
\label{fielddesc:I2CRXFIFOOVFINTCLR}\item [I2C\_RXFIFO\_OVF\_INT\_CLR] 置位此位,清除 I2C\_RXFIFO\_OVF\_INT 中断。 (WT)
\label{fielddesc:I2CENDDETECTINTCLR}\item [I2C\_END\_DETECT\_INT\_CLR] 置位此位,清除 I2C\_END\_DETECT\_INT 中断。 (WT)
\label{fielddesc:I2CBYTETRANSDONEINTCLR}\item [I2C\_BYTE\_TRANS\_DONE\_INT\_CLR] 置位此位,清除 I2C\_BYTE\_TRANS\_DONE\_INT 中断。 (WT)
\label{fielddesc:I2CARBITRATIONLOSTINTCLR}\item [I2C\_ARBITRATION\_LOST\_INT\_CLR] 置位此位,清除 I2C\_ARBITRATION\_LOST\_INT 中断。 (WT)
\label{fielddesc:I2CMSTTXFIFOUDFINTCLR}\item [I2C\_MST\_TXFIFO\_UDF\_INT\_CLR] 置位此位,清除 I2C\_MST\_TXFIFO\_UDF\_INT 中断。 (WT)
\label{fielddesc:I2CTRANSCOMPLETEINTCLR}\item [I2C\_TRANS\_COMPLETE\_INT\_CLR] 置位此位,清除 I2C\_TRANS\_COMPLETE\_INT 中断。 (WT)
\label{fielddesc:I2CTIMEOUTINTCLR}\item [I2C\_TIME\_OUT\_INT\_CLR] 置位此位,清除 I2C\_TIME\_OUT\_INT 中断。 (WT)
\label{fielddesc:I2CTRANSSTARTINTCLR}\item [I2C\_TRANS\_START\_INT\_CLR] 置位此位,清除 I2C\_TRANS\_START\_INT 中断。 (WT)
\label{fielddesc:I2CNACKINTCLR}\item [I2C\_NACK\_INT\_CLR] 置位此位,清除 I2C\_NACK\_INT 中断。 (WT)
\label{fielddesc:I2CTXFIFOOVFINTCLR}\item [I2C\_TXFIFO\_OVF\_INT\_CLR] 置位此位,清除 I2C\_TXFIFO\_OVF\_INT 中断。 (WT)
\label{fielddesc:I2CRXFIFOUDFINTCLR}\item [I2C\_RXFIFO\_UDF\_INT\_CLR] 置位此位,清除 I2C\_RXFIFO\_UDF\_INT  中断。 (WT)
\label{fielddesc:I2CSCLSTTOINTCLR}\item [I2C\_SCL\_ST\_TO\_INT\_CLR] 置位此位,清除 I2C\_SCL\_ST\_TO\_INT 中断。 (WT)
\label{fielddesc:I2CSCLMAINSTTOINTCLR}\item [I2C\_SCL\_MAIN\_ST\_TO\_INT\_CLR] 置位此位,清除 I2C\_SCL\_MAIN\_ST\_TO\_INT 中断。 (WT)
\label{fielddesc:I2CDETSTARTINTCLR}\item [I2C\_DET\_START\_INT\_CLR] 置位此位,清除 I2C\_DET\_START\_INT 中断。 (WT)
\label{fielddesc:I2CSLAVESTRETCHINTCLR}\item [I2C\_SLAVE\_STRETCH\_INT\_CLR] 置位此位,清除 I2C\_SLAVE\_STRETCH\_INT 中断。 (WT)
\label{fielddesc:I2CGENERALCALLINTCLR}\item [I2C\_GENERAL\_CALL\_INT\_CLR] 置位此位,清除 I2C\_GENARAL\_CALL\_INT 中断。 (WT)
\end{reglist}\end{regdesc}
\end{register}


\begin{register}{H}{I2C\_INT\_ENA\_REG}{0x{}0028}\label{regdesc:I2CINTENAREG}
\regfield{(reserved)}{14}{18}{00000000000000}%
\regfield{I2C\_GENERAL\_CALL\_INT\_ENA}{1}{17}{{0}}%
\regfield{I2C\_SLAVE\_STRETCH\_INT\_ENA}{1}{16}{{0}}%
\regfield{I2C\_DET\_START\_INT\_ENA}{1}{15}{{0}}%
\regfield{I2C\_SCL\_MAIN\_ST\_TO\_INT\_ENA}{1}{14}{{0}}%
\regfield{I2C\_SCL\_ST\_TO\_INT\_ENA}{1}{13}{{0}}%
\regfield{I2C\_RXFIFO\_UDF\_INT\_ENA}{1}{12}{{0}}%
\regfield{I2C\_TXFIFO\_OVF\_INT\_ENA}{1}{11}{{0}}%
\regfield{I2C\_NACK\_INT\_ENA}{1}{10}{{0}}%
\regfield{I2C\_TRANS\_START\_INT\_ENA}{1}{9}{{0}}%
\regfield{I2C\_TIME\_OUT\_INT\_ENA}{1}{8}{{0}}%
\regfield{I2C\_TRANS\_COMPLETE\_INT\_ENA}{1}{7}{{0}}%
\regfield{I2C\_MST\_TXFIFO\_UDF\_INT\_ENA}{1}{6}{{0}}%
\regfield{I2C\_ARBITRATION\_LOST\_INT\_ENA}{1}{5}{{0}}%
\regfield{I2C\_BYTE\_TRANS\_DONE\_INT\_ENA}{1}{4}{{0}}%
\regfield{I2C\_END\_DETECT\_INT\_ENA}{1}{3}{{0}}%
\regfield{I2C\_RXFIFO\_OVF\_INT\_ENA}{1}{2}{{0}}%
\regfield{I2C\_TXFIFO\_WM\_INT\_ENA}{1}{1}{{0}}%
\regfield{I2C\_RXFIFO\_WM\_INT\_ENA}{1}{0}{{0}}%
\reglabel{Reset}\regnewline%
\begin{regdesc}\begin{reglist}
\label{fielddesc:I2CRXFIFOWMINTENA}\item [I2C\_RXFIFO\_WM\_INT\_ENA] I2C\_RXFIFO\_WM\_INT 的使能位。 (R/W)
\label{fielddesc:I2CTXFIFOWMINTENA}\item [I2C\_TXFIFO\_WM\_INT\_ENA] I2C\_TXFIFO\_WM\_INT 的使能位。 (R/W)
\label{fielddesc:I2CRXFIFOOVFINTENA}\item [I2C\_RXFIFO\_OVF\_INT\_ENA] I2C\_RXFIFO\_OVF\_INT 的使能位。 (R/W)
\label{fielddesc:I2CENDDETECTINTENA}\item [I2C\_END\_DETECT\_INT\_ENA] I2C\_END\_DETECT\_INT 的使能位。 (R/W)
\label{fielddesc:I2CBYTETRANSDONEINTENA}\item [I2C\_BYTE\_TRANS\_DONE\_INT\_ENA] I2C\_BYTE\_TRANS\_DONE\_INT 的使能位。 (R/W)
\label{fielddesc:I2CARBITRATIONLOSTINTENA}\item [I2C\_ARBITRATION\_LOST\_INT\_ENA] I2C\_ARBITRATION\_LOST\_INT 的使能位。 (R/W)
\label{fielddesc:I2CMSTTXFIFOUDFINTENA}\item [I2C\_MST\_TXFIFO\_UDF\_INT\_ENA] I2C\_MST\_TXFIFO\_UDF\_INT 的使能位。 (R/W)
\label{fielddesc:I2CTRANSCOMPLETEINTENA}\item [I2C\_TRANS\_COMPLETE\_INT\_ENA] I2C\_TRANS\_COMPLETE\_INT 的使能位。 (R/W)
\label{fielddesc:I2CTIMEOUTINTENA}\item [I2C\_TIME\_OUT\_INT\_ENA] I2C\_TIME\_OUT\_INT 的使能位。 (R/W)
\label{fielddesc:I2CTRANSSTARTINTENA}\item [I2C\_TRANS\_START\_INT\_ENA] I2C\_TRANS\_START\_INT 的使能位。 (R/W)
\label{fielddesc:I2CNACKINTENA}\item [I2C\_NACK\_INT\_ENA] I2C\_NACK\_INT 的使能位。 (R/W)
\label{fielddesc:I2CTXFIFOOVFINTENA}\item [I2C\_TXFIFO\_OVF\_INT\_ENA] I2C\_TXFIFO\_OVF\_INT 的使能位。 (R/W)
\label{fielddesc:I2CRXFIFOUDFINTENA}\item [I2C\_RXFIFO\_UDF\_INT\_ENA] I2C\_RXFIFO\_UDF\_INT 的使能位。 (R/W)
\label{fielddesc:I2CSCLSTTOINTENA}\item [I2C\_SCL\_ST\_TO\_INT\_ENA] I2C\_SCL\_ST\_TO\_INT 的使能位。 (R/W)
\label{fielddesc:I2CSCLMAINSTTOINTENA}\item [I2C\_SCL\_MAIN\_ST\_TO\_INT\_ENA] I2C\_SCL\_MAIN\_ST\_TO\_INT 的使能位。 (R/W)
\label{fielddesc:I2CDETSTARTINTENA}\item [I2C\_DET\_START\_INT\_ENA] I2C\_DET\_START\_INT 的使能位。 (R/W)
\label{fielddesc:I2CSLAVESTRETCHINTENA}\item [I2C\_SLAVE\_STRETCH\_INT\_ENA] I2C\_SLAVE\_STRETCH\_INT 的使能位。 (R/W)
\label{fielddesc:I2CGENERALCALLINTENA}\item [I2C\_GENERAL\_CALL\_INT\_ENA] I2C\_GENARAL\_CALL\_INT 的使能位。 (R/W)
\end{reglist}\end{regdesc}
\end{register}


\begin{register}{H}{I2C\_INT\_STATUS\_REG}{0x{}002C}\label{regdesc:I2CINTSTATUSREG}
\regfield{(reserved)}{14}{18}{00000000000000}%
\regfield{I2C\_GENERAL\_CALL\_INT\_ST}{1}{17}{{0}}%
\regfield{I2C\_SLAVE\_STRETCH\_INT\_ST}{1}{16}{{0}}%
\regfield{I2C\_DET\_START\_INT\_ST}{1}{15}{{0}}%
\regfield{I2C\_SCL\_MAIN\_ST\_TO\_INT\_ST}{1}{14}{{0}}%
\regfield{I2C\_SCL\_ST\_TO\_INT\_ST}{1}{13}{{0}}%
\regfield{I2C\_RXFIFO\_UDF\_INT\_ST}{1}{12}{{0}}%
\regfield{I2C\_TXFIFO\_OVF\_INT\_ST}{1}{11}{{0}}%
\regfield{I2C\_NACK\_INT\_ST}{1}{10}{{0}}%
\regfield{I2C\_TRANS\_START\_INT\_ST}{1}{9}{{0}}%
\regfield{I2C\_TIME\_OUT\_INT\_ST}{1}{8}{{0}}%
\regfield{I2C\_TRANS\_COMPLETE\_INT\_ST}{1}{7}{{0}}%
\regfield{I2C\_MST\_TXFIFO\_UDF\_INT\_ST}{1}{6}{{0}}%
\regfield{I2C\_ARBITRATION\_LOST\_INT\_ST}{1}{5}{{0}}%
\regfield{I2C\_BYTE\_TRANS\_DONE\_INT\_ST}{1}{4}{{0}}%
\regfield{I2C\_END\_DETECT\_INT\_ST}{1}{3}{{0}}%
\regfield{I2C\_RXFIFO\_OVF\_INT\_ST}{1}{2}{{0}}%
\regfield{I2C\_TXFIFO\_WM\_INT\_ST}{1}{1}{{0}}%
\regfield{I2C\_RXFIFO\_WM\_INT\_ST}{1}{0}{{0}}%
\reglabel{Reset}\regnewline%
\begin{regdesc}\begin{reglist}
\label{fielddesc:I2CRXFIFOWMINTST}\item [I2C\_RXFIFO\_WM\_INT\_ST] I2C\_RXFIFO\_WM\_INT 的屏蔽状态位。 (RO)
\label{fielddesc:I2CTXFIFOWMINTST}\item [I2C\_TXFIFO\_WM\_INT\_ST] I2C\_TXFIFO\_WM\_INT 的屏蔽状态位。 (RO)
\label{fielddesc:I2CRXFIFOOVFINTST}\item [I2C\_RXFIFO\_OVF\_INT\_ST] I2C\_RXFIFO\_OVF\_INT 的屏蔽状态位。 (RO)
\label{fielddesc:I2CENDDETECTINTST}\item [I2C\_END\_DETECT\_INT\_ST] I2C\_END\_DETECT\_INT 的屏蔽状态位。 (RO)
\label{fielddesc:I2CBYTETRANSDONEINTST}\item [I2C\_BYTE\_TRANS\_DONE\_INT\_ST] I2C\_BYTE\_TRANS\_DONE\_INT 的 屏蔽状态位。 (RO)
\label{fielddesc:I2CARBITRATIONLOSTINTST}\item [I2C\_ARBITRATION\_LOST\_INT\_ST] I2C\_ARBITRATION\_LOST\_INT 的屏蔽状态位。 (RO)
\label{fielddesc:I2CMSTTXFIFOUDFINTST}\item [I2C\_MST\_TXFIFO\_UDF\_INT\_ST] I2C\_MST\_TXFIFO\_UDF\_INT 的屏蔽状态位。 (RO)
\label{fielddesc:I2CTRANSCOMPLETEINTST}\item [I2C\_TRANS\_COMPLETE\_INT\_ST] I2C\_TRANS\_COMPLETE\_INT 的屏蔽状态位。 (RO)
\label{fielddesc:I2CTIMEOUTINTST}\item [I2C\_TIME\_OUT\_INT\_ST] I2C\_TIME\_OUT\_INT 的屏蔽状态位。 (RO)
\label{fielddesc:I2CTRANSSTARTINTST}\item [I2C\_TRANS\_START\_INT\_ST] I2C\_TRANS\_START\_INT 的屏蔽状态位。 (RO)
\label{fielddesc:I2CNACKINTST}\item [I2C\_NACK\_INT\_ST] I2C\_NACK\_INT 的屏蔽状态位。 (RO)
\label{fielddesc:I2CTXFIFOOVFINTST}\item [I2C\_TXFIFO\_OVF\_INT\_ST] I2C\_TXFIFO\_OVF\_INT 的屏蔽状态位。 (RO)
\label{fielddesc:I2CRXFIFOUDFINTST}\item [I2C\_RXFIFO\_UDF\_INT\_ST] I2C\_RXFIFO\_UDF\_INT 的屏蔽状态位。 (RO)
\label{fielddesc:I2CSCLSTTOINTST}\item [I2C\_SCL\_ST\_TO\_INT\_ST] I2C\_SCL\_ST\_TO\_INT 的屏蔽状态位。 (RO)
\label{fielddesc:I2CSCLMAINSTTOINTST}\item [I2C\_SCL\_MAIN\_ST\_TO\_INT\_ST] I2C\_SCL\_MAIN\_ST\_TO\_INT 的屏蔽状态位。 (RO)
\label{fielddesc:I2CDETSTARTINTST}\item [I2C\_DET\_START\_INT\_ST] I2C\_DET\_START\_INT 的屏蔽状态位。 (RO)
\label{fielddesc:I2CSLAVESTRETCHINTST}\item [I2C\_SLAVE\_STRETCH\_INT\_ST] I2C\_SLAVE\_STRETCH\_INT 的屏蔽状态位。 (RO)
\label{fielddesc:I2CGENERALCALLINTST}\item [I2C\_GENERAL\_CALL\_INT\_ST] I2C\_GENARAL\_CALL\_INT 的屏蔽状态位。 (RO)
\end{reglist}\end{regdesc}
\end{register}


\begin{register}{H}{I2C\_COMD0\_REG}{0x{}0058}\label{regdesc:I2CCOMD0REG}
\regfield{I2C\_COMMAND0\_DONE}{1}{31}{{0}}%
\regfield{(reserved)}{17}{14}{00000000000000000}%
\regfield{I2C\_COMMAND0}{14}{0}{{0}}%
\reglabel{Reset}\regnewline%
\begin{regdesc}\begin{reglist}
\label{fielddesc:I2CCOMMAND0}\item [I2C\_COMMAND0] 命令寄存器 0 的内容。 该命令包括三个部分:
\begin{itemize}
    \item op\_code 为命令,1:WRITE;2:STOP;3:READ;4:END;6:RSTART。
    \item byte\_num 表示需发送或接收的字节数。
    \item ack\_check\_en、ack\_exp 和 ack 用于控制 ACK 位。 更多信息详见章节 \ref{sec:i2c-func-descr-cmd-controller}。
\end{itemize} (R/W)
\label{fielddesc:I2CCOMMAND0DONE}\item [I2C\_COMMAND0\_DONE] 在 I2C 主机模式下完成命令 0 时,该位翻转为高电平。 (R/W/SS)
\end{reglist}\end{regdesc}
\end{register}


\begin{register}{H}{I2C\_COMD1\_REG}{0x{}005C}\label{regdesc:I2CCOMD1REG}
\regfield{I2C\_COMMAND1\_DONE}{1}{31}{{0}}%
\regfield{(reserved)}{17}{14}{00000000000000000}%
\regfield{I2C\_COMMAND1}{14}{0}{{0}}%
\reglabel{Reset}\regnewline%
\begin{regdesc}\begin{reglist}
\label{fielddesc:I2CCOMMAND1}\item [I2C\_COMMAND1] 命令寄存器 1 的内容,同 \hyperref[fielddesc:I2CCOMMAND0]{I2C\_COMMAND0}。 (R/W)
\label{fielddesc:I2CCOMMAND1DONE}\item [I2C\_COMMAND1\_DONE] 在 I2C 主机模式下完成命令 1 时,该位翻转为高电平。 (R/W/SS)
\end{reglist}\end{regdesc}
\end{register}


\begin{register}{H}{I2C\_COMD2\_REG}{0x{}0060}\label{regdesc:I2CCOMD2REG}
\regfield{I2C\_COMMAND2\_DONE}{1}{31}{{0}}%
\regfield{(reserved)}{17}{14}{00000000000000000}%
\regfield{I2C\_COMMAND2}{14}{0}{{0}}%
\reglabel{Reset}\regnewline%
\begin{regdesc}\begin{reglist}
\label{fielddesc:I2CCOMMAND2}\item [I2C\_COMMAND2] 命令寄存器 2 的内容,同 \hyperref[fielddesc:I2CCOMMAND0]{I2C\_COMMAND0}。 (R/W)
\label{fielddesc:I2CCOMMAND2DONE}\item [I2C\_COMMAND2\_DONE] 在 I2C 主机模式下完成命令 2 时,该位翻转为高电平。 (R/W/SS)
\end{reglist}\end{regdesc}
\end{register}


\begin{register}{H}{I2C\_COMD3\_REG}{0x{}0064}\label{regdesc:I2CCOMD3REG}
\regfield{I2C\_COMMAND3\_DONE}{1}{31}{{0}}%
\regfield{(reserved)}{17}{14}{00000000000000000}%
\regfield{I2C\_COMMAND3}{14}{0}{{0}}%
\reglabel{Reset}\regnewline%
\begin{regdesc}\begin{reglist}
\label{fielddesc:I2CCOMMAND3}\item [I2C\_COMMAND3] 命令寄存器 3 的内容,同 \hyperref[fielddesc:I2CCOMMAND0]{I2C\_COMMAND0}。 (R/W)
\label{fielddesc:I2CCOMMAND3DONE}\item [I2C\_COMMAND3\_DONE] 在 I2C 主机模式下完成命令 3 时,该位翻转为高电平。 (R/W/SS)
\end{reglist}\end{regdesc}
\end{register}


\begin{register}{H}{I2C\_COMD4\_REG}{0x{}0068}\label{regdesc:I2CCOMD4REG}
\regfield{I2C\_COMMAND4\_DONE}{1}{31}{{0}}%
\regfield{(reserved)}{17}{14}{00000000000000000}%
\regfield{I2C\_COMMAND4}{14}{0}{{0}}%
\reglabel{Reset}\regnewline%
\begin{regdesc}\begin{reglist}
\label{fielddesc:I2CCOMMAND4}\item [I2C\_COMMAND4] 命令寄存器 4 的内容,同 \hyperref[fielddesc:I2CCOMMAND0]{I2C\_COMMAND0}。 (R/W)
\label{fielddesc:I2CCOMMAND4DONE}\item [I2C\_COMMAND4\_DONE] 在 I2C 主机模式下完成命令 4 时,该位翻转为高电平。 (R/W/SS)
\end{reglist}\end{regdesc}
\end{register}


\begin{register}{H}{I2C\_COMD5\_REG}{0x{}006C}\label{regdesc:I2CCOMD5REG}
\regfield{I2C\_COMMAND5\_DONE}{1}{31}{{0}}%
\regfield{(reserved)}{17}{14}{00000000000000000}%
\regfield{I2C\_COMMAND5}{14}{0}{{0}}%
\reglabel{Reset}\regnewline%
\begin{regdesc}\begin{reglist}
\label{fielddesc:I2CCOMMAND5}\item [I2C\_COMMAND5] 命令寄存器 5 的内容,同 \hyperref[fielddesc:I2CCOMMAND0]{I2C\_COMMAND0}。 (R/W)
\label{fielddesc:I2CCOMMAND5DONE}\item [I2C\_COMMAND5\_DONE] 在 I2C 主机模式下完成命令 5 时,该位翻转为高电平。 (R/W/SS)
\end{reglist}\end{regdesc}
\end{register}


\begin{register}{H}{I2C\_COMD6\_REG}{0x{}0070}\label{regdesc:I2CCOMD6REG}
\regfield{I2C\_COMMAND6\_DONE}{1}{31}{{0}}%
\regfield{(reserved)}{17}{14}{00000000000000000}%
\regfield{I2C\_COMMAND6}{14}{0}{{0}}%
\reglabel{Reset}\regnewline%
\begin{regdesc}\begin{reglist}
\label{fielddesc:I2CCOMMAND6}\item [I2C\_COMMAND6] 命令寄存器 6 的内容,同 \hyperref[fielddesc:I2CCOMMAND0]{I2C\_COMMAND0}。 (R/W)
\label{fielddesc:I2CCOMMAND6DONE}\item [I2C\_COMMAND6\_DONE] 在 I2C 主机模式下完成命令 6 时,该位翻转为高电平。 (R/W/SS)
\end{reglist}\end{regdesc}
\end{register}


\begin{register}{H}{I2C\_COMD7\_REG}{0x{}0074}\label{regdesc:I2CCOMD7REG}
\regfield{I2C\_COMMAND7\_DONE}{1}{31}{{0}}%
\regfield{(reserved)}{17}{14}{00000000000000000}%
\regfield{I2C\_COMMAND7}{14}{0}{{0}}%
\reglabel{Reset}\regnewline%
\begin{regdesc}\begin{reglist}
\label{fielddesc:I2CCOMMAND7}\item [I2C\_COMMAND7] 命令寄存器 7 的内容,同 \hyperref[fielddesc:I2CCOMMAND0]{I2C\_COMMAND0}。 (R/W)
\label{fielddesc:I2CCOMMAND7DONE}\item [I2C\_COMMAND7\_DONE] 在 I2C 主机模式下完成命令 7 时,该位翻转为高电平。 (R/W/SS)
\end{reglist}\end{regdesc}
\end{register}


\begin{register}{H}{I2C\_DATE\_REG}{0x{}00F8}\label{regdesc:I2CDATEREG}
\regfield{I2C\_DATE}{32}{0}{{0x{}20070201}}%
\reglabel{Reset}\regnewline%
\begin{regdesc}\begin{reglist}
\label{fielddesc:I2CDATE}\item [I2C\_DATE] 版本控制寄存器。 (R/W)
\end{reglist}\end{regdesc}
\end{register}
