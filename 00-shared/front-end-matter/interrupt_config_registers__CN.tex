\phantomsection
\chapter*{中断配置寄存器}\label{interrupt-config-registers}
\markboth{\it{中断配置寄存器}}{\Navigation}
\addcontentsline{toc}{chapter}{中断配置寄存器}

大部分外设的内部中断源都有以下配置寄存器:

\begin{itemize}
    \item \textbf{RAW}(原始状态)寄存器:该寄存器指示原始中断状态,每个位对应一个内部中断源。当中断源触发时,其 RAW 位为 1。
    
    \item \textbf{ENA}(使能)寄存器:该寄存器用于启用或禁用内部中断源,每个位对应一个内部中断源。
    
    通过操作 ENA 寄存器,可以根据需要屏蔽或取消屏蔽某个内部中断源。当中断源被屏蔽(禁用)时,它不会生成中断信号,但仍可以从 RAW 寄存器中读取其值。
    
    \item \textbf{ST}(状态)寄存器:该寄存器指示中断源的屏蔽状态,每个位对应一个内部中断源。ST 位为 1 代表 RAW 位和 ENA 位都为 1,即中断源已生成且未被屏蔽。RAW 位和 ENA 位的值为其他组合时,ST 位为 0。
    
    ENA/RAW/ST 寄存器的配置见表 \ref{tab:int-raw-st-ena}。

    \item \textbf{CLR}(清除)寄存器:CLR 寄存器负责清除内部中断源。写 1 将清除该位对应的中断源。
\end{itemize}

\begin{longtable}{ | l | l | l | }
\caption{ENA/RAW/ST 寄存器的配置} \label{tab:int-raw-st-ena}
\\ \hline
\rowcolor{lightgray}
\textbf{ENA 位的值} & \textbf{RAW 位的值}  &  \textbf{ST 位的值}\\ \hline
0 & 忽略 & 0 \\ \hline
\multirow{2}*{1} &  0 & 0 \\ \cline{2-3}
 &  1 & 1 \\ \hline
\end{longtable}