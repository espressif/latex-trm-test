\phantomsection
\chapter*{Programming Reserved Register Field}\label{programming-reserved-field}
\markboth{\it{Programming Reserved Register Field}}{\Navigation}
\addcontentsline{toc}{chapter}{Programming Reserved Register Field}

\phantomsection
\section*{Introduction}
\addcontentsline{toc}{section}{Introduction}

A field in a register is reserved if the field is not open to users, or produces unpredictable results if configured to values other than defaults.

\phantomsection
\section*{Programming Reserved Register Field}
\addcontentsline{toc}{section}{Programming Reserved Register Field}

The reserved fields should not be modified. It is not possible to write only part of a register since registers must always be written as a whole. As a result, to write an entire register that contains reserved fields, you can choose one of the following two options:

\begin{enumerate}
    \item Read the value of the register, modify only the fields you want to configure and then write back the value so that reserved fields are untouched.

\noindent OR

\item Modify only the fields you want to configure and write back the default value of the reserved fields. The default value of a field is provided in the "Reset" line of a register diagram. For example, the default value of Field\_A in \hyperref[fig:registerX]{Register X} is 1.
\end{enumerate}

\begin{register}{H}{Register X}{Address}\label{fig:registerX}
\regfield{(reserved)}{12}{20}{000000000000}%
\regfield{Field\_C}{4}{16}{{0000}}%
\regfield{(reserved)}{14}{2}{00000000000000}%
\regfield{Field\_B}{1}{1}{{0}}%
\regfield{Field\_A}{1}{0}{{1}}%
\reglabel{Reset}\regnewline%

\end{register}

Suppose you want to set Field\_A, Field\_B, and Field\_C of \hyperref[fig:registerX]{Register X} to 0x0, 0x1, and 0x2, you can:

\begin{itemize}
    \item Use option 1 and fill in the reserved fields with the value you have just read. Suppose the register reads as 0x0000\_0003. Then, you can modify the fields you want to configure, thus writing 0x0002\_0002 to the register.
    \item Use option 2 and fill in the reserved fields with their defaults, thus writing 0x0002\_0002 to the register.
\end{itemize}
