\phantomsection
\chapter*{如何配置寄存器的保留域}\label{programming-reserved-field}
\markboth{\it{如何配置寄存器的保留域}}{\Navigation}
\addcontentsline{toc}{chapter}{如何配置寄存器的保留域}

\phantomsection
\section*{概述}
\addcontentsline{toc}{section}{概述}

寄存器的保留域指的是寄存器中不对用户开放、或者配置为非默认值时会导致不可预测的结果的域。

\phantomsection
\section*{如何配置保留域}
\addcontentsline{toc}{section}{如何配置保留域}

保留域的值不能修改。由于写寄存器时必须整体写,不能只写部分域,因此,在写带有保留域的寄存器时需要特别注意,只能采用以下两种方式:

\begin{enumerate}
\item 读取寄存器的值,仅修改需要配置的域,然后将修改后的值以及其他未修改的值一起写回寄存器,这样保留域的值就会保持不变。

\noindent 或者

\item 仅修改需要配置的域,然后将保留域的默认值写回寄存器。默认值即寄存器图表中的“Reset”值。例如,\hyperref[fig:registerX]{寄存器 X} 中 Field\_A 的默认值为 1。
\end{enumerate}

\begin{register}{H}{寄存器 X}{地址}\label{fig:registerX}
\regfield{(reserved)}{12}{20}{000000000000}%
\regfield{Field\_C}{4}{16}{{0000}}%
\regfield{(reserved)}{14}{2}{00000000000000}%
\regfield{Field\_B}{1}{1}{{0}}%
\regfield{Field\_A}{1}{0}{{1}}%
\reglabel{Reset}\regnewline%

\end{register}

假设您要将 \hyperref[fig:registerX]{寄存器 X} 的 Field\_A、Field\_B 和 Field\_C 设置为 0x0、0x1 和 0x2,您可以:

\begin{itemize}
\item 使用第 1 种方式,修改这三个域的值,然后把读到的值写回寄存器。假设寄存器读取的值为 0x0000\_0003,修改三个域的值后,将值 0x0002\_0002 写入寄存器。
\item 使用第 2 种方式,修改这三个域的值,然后把保留域的默认值写回寄存器,即把 0x0002\_0002 写入寄存器。
\end{itemize}
