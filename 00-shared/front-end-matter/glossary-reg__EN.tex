%%% Abbreviations Related to Registers %%%

\phantomsection
\section*{Abbreviations Related to Registers}\label{glossary-abbr-reg}
\addcontentsline{toc}{section}{Abbreviations Related to Registers}

\begin{longtable}[c]{ R{3cm} p{12cm} }

REG     & \textbf{Register}. \\
SYSREG  & \textbf{System registers} are a group of registers that control system reset, memory, clocks, software interrupts, power management, clock gating, etc. \\
ISO     & \textbf{Isolation}. If a peripheral or other chip component is powered down, the pins, if any, to which its output signals are routed will go into a floating state. ISO registers isolate such pins and keep them at a certain determined value, so that the other non-powered-down peripherals/devices attached to these pins are not affected. \\
NMI     & \textbf{Non-maskable interrupt} is a hardware interrupt that cannot be disabled or ignored by the CPU instructions. Such interrupts exist to signal the occurrence of a critical error. \\
W1TS    & Abbreviation added to names of registers/fields to indicate that such register/field should be used to set a field in a corresponding register with a similar name. For example, the register \texttt{GPIO\_ENABLE\_W1TS\_REG} should be used to set the corresponding fields in the register \texttt{GPIO\_ENABLE\_REG}. \\
W1TC    & Same as \textit{W1TS}, but used to clear a field in a corresponding register. \\
% FPU   &
% FPD   &
% PLLA  &
% BBPLL &

\end{longtable}


%%% Access Types for Registers %%%

% \clearpage
\phantomsection
\section*{Access Types for Registers}\label{glossary-access-types}
\addcontentsline{toc}{section}{Access Types for Registers}

Sections \textit{Register Summary} and \textit{Register Description} in TRM chapters specify access types for registers and their fields.

Most frequently used access types and their combinations are as follows:

\begin{multicols}{4}
  \begin{itemize}
    \item RO
    \item WO
    \item WT
    \item R/W
    \item WL
    \item R/W/SC
    \item R/W/SS
    \item R/W/SS/SC
    \item R/WC/SS
    \item R/WC/SC
    \item R/WC/SS/SC
    \item R/WS/SC
    \item R/WS/SS
    \item R/WS/SS/SC
    \item R/SS/WTC
    \item R/SC/WTC
    \item R/SS/SC/WTC
    \item RF/WF
    \item R/SS/RC
    \item varies
  \end{itemize}
\end{multicols}

Descriptions of all access types are provided below.


\begin{longtable}[c]{ R{1cm} p{13cm} }

% table contents
R & \textbf{Read.} 
	User application can read from this register/field; usually combined with other access types. \\
RO & \textbf{Read only.} 
	User application can only read from this register/field. \\
HRO & \textbf{Hardware Read Only.} 
	Only hardware can read from this register/field; used for storing default settings for variable parameters. \\
W & \textbf{Write.} 
	User application can write to this register/field; usually combined with other access types. \\
WO & \textbf{Write only.} 
	User application can only write to this register/field. \\

SS & \textbf{Self set.} 
	On a specified event, hardware automatically writes 1 to this register/field; used with 1-bit fields. \\
SC & \textbf{Self clear.} 
	On a specified event, hardware automatically writes 0 to this register/field; used with 1-bit and multi-bit fields. \\
SM & \textbf{Self modify.} 
	On a specified event, hardware automatically writes a specified value to this register/field; used with multi-bit fields. \\
SU & \textbf{Self update.} 
	On a specified event, hardware automatically updates this register/field; used with multi-bit fields. \\

RS & \textbf{Read to set.} 
	If user application reads from this register/field, hardware automatically writes 1 to it. \\
RC & \textbf{Read to clear.} 
	If user application reads from this register/field, hardware automatically writes 0 to it. \\
RF & \textbf{Read from FIFO.} 
	If user application writes new data to FIFO, the register/field automatically reads it. \\

WF & \textbf{Write to FIFO.} 
	If user application writes new data to this register/field, it automatically passes the data to FIFO via APB bus. \\

WS & \textbf{Write any value to set.} 
	If user application writes to this register/field, hardware automatically sets this register/field. \\
W1S & \textbf{Write 1 to set.} 
	If user application writes 1 to this register/field, hardware automatically sets this register/field. \\
W0S & \textbf{Write 0 to set.} 
	If user application writes 0 to this register/field, hardware automatically sets this register/field. \\

WC & \textbf{Write any value to clear.} 
	If user application writes to this register/field, hardware automatically clears this register/field. \\
W1C & \textbf{Write 1 to clear.} 
	If user application writes 1 to this register/field, hardware automatically clears this register/field. \\
W0C & \textbf{Write 0 to clear.} 
	If user application writes 0 to this register/field, hardware automatically clears this register/field. \\

WT & \textbf{Write 1 to trigger an event.} 
	If user application writes 1 to this field, this action triggers an event (pulse in the APB bus) or clears a corresponding WTC field (see WTC). \\
WTC & \textbf{Write to clear.} 
	Hardware automatically clears this field if user application writes 1 to the corresponding WT field (see WT). \\

W1T & \textbf{Write 1 to toggle.} 
	If user application writes 1 to this field, hardware automatically inverts the corresponding field; otherwise - no effect. \\
W0T & \textbf{Write 0 to toggle.} 
	If user application writes 0 to this field, hardware automatically inverts the corresponding field; otherwise - no effect. \\

WL & \textbf{Write if a lock is deactivated.} 
	If the lock is deactivated, user application can write to this register/field. \\

varies & \textbf{The access type varies.} 
	 Different fields of this register might have different access types. \\

\end{longtable}
